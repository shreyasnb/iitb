\documentclass[12pt]{article}
\usepackage[margin=3cm]{geometry}

\usepackage{graphicx}
\usepackage{array}
\usepackage{url}
\usepackage{multicol}
\usepackage{hyperref}
\usepackage{amsmath}
\usepackage{amssymb}

\begin{document}

\begin{figure}
    \centering
    \includegraphics[width=5cm]{Images/logo.png}
\end{figure}

\begin{center}
    \textbf{\LARGE Indian Institute of Technology Bombay} \\
    \vspace{1cm}
    \vspace{0.3cm}

    \rule{\linewidth}{0.5pt} \\
    \vspace{0.2cm}
    \textbf{\LARGE AE 6103 \\ \vspace{0.3cm} Introduction to Space Technology} \\
    \vspace{0.1cm}
    \rule{\linewidth}{0.5pt} \\
    \vspace{1.5cm}
    \LARGE Global Space Industry: Present and Future\\

    \vspace{2cm}

    \normalsize Shreyas N B \\
    210010061

    \vspace{3cm}

    \normalsize\textit{Course Instructor:}\vspace{0.2cm}
    Prof.\ T. Chandra Sekar

    \vspace{1cm}
    \date{}
\end{center}

\newpage

\tableofcontents

\newpage

\section{Introduction}
The Global Space Industry stands as a testament to humanity's insatiable curiosity and relentless pursuit of exploration. From the launch of Sputnik in 1957 to the recent Mars rover missions, space exploration has captivated the imagination of people around the world, inspiring scientific discovery, technological innovation, and international cooperation. Today, the Global Space Industry encompasses a diverse array of stakeholders, including government space agencies, private companies, and research institutions, all of which contribute to the advancement of space exploration, satellite technology, and commercial space activities.

\begin{figure}[ht]
    \centering
    \includegraphics[width=0.8\textwidth]{Images/space-report.jpg}
    \caption{Key Factors in the Global Space Industry}
\end{figure}

\section{Literature/Market Survey}

\subsection{Space Economy - A Definition}

The following working definition forms a starting point for the analysis of the space economy:
\begin{quote}
    \textit{The space economy is the full range of activities and the use of resources that create and provide value and benefits to human beings in the course of exploring, understanding, managing and utilising space.}
\end{quote}

Hence, it includes all public and private actors involved in developing, providing and using space-related products and services, ranging from research and development, the manufacture and use of space infrastructure (ground stations, launch vehicles and satellites) to space-enabled applications (navigation equipment, satellite phones, meteorological services, etc.) and the scientific knowledge generated by such activities. 

Table \ref{tab:estimates} contains estimates of the size of the space economy taken from various recent publications. According to some financial estimates, the space economy may surpass USD 1 trillion by 2040. The estimates are based on different methodologies and assumptions, and the differences between them are significant. The great discrepancy in estimates is therefore largely due to the use of different definitions and delimitations of the space economy. In particular, the inclusion or exclusion of services supporting
consumer markets such as direct-to-home television, and consumer applications relying on global navigation satellite systems (GNSS) signals, etc.

\begin{table}[ht]
    \centering
    \begin{tabular}{|c|c|c|}
        \hline
        \textbf{Organisation} & \textbf{Recent Estimates (2016)} & \textbf{Forecasts (2040)} \\
        \hline \hline
        Satellite Industry Association & USD 339.1 billion & n.a. \\
        \hline
        Morgan Stanley & USD 350 billion & USD 1.1 trillion \\
        \hline
        Merrill Lynch/Bank of America & USD 350 billion & USD 2.7 trillion  \\
        \hline
        Space Foundation & USD 329.3 billion & n.a. \\
        \hline
        Institute for Defense Analyses & USD 166.8 billion & n.a. \\
        \hline
    \end{tabular}
    \caption{Source: OECD}
    \label{tab:estimates}
\end{table}

\subsection{Role of the government sector in space economy}

The government sector plays a key role in the space economy as investor, developer, owner, operator, regulator and customer. National agencies, research centres and laboratories also perform space R\&D and, in some cases, have a manufacturing role (e.g. India, Korea). The bulk of their funding tends to be public, but they may also receive private financing via contracts and licensing arrangements etc. The international classification of actors involved in R\&D, as described in the Frascati Manual, is often used to gather comparable data concerning the R\&D activities of governments.

\begin{figure}[ht]
    \centering
    \includegraphics[width=0.8\textwidth]{Images/space-gdp.png}
    \caption{Measured as a share of GDP in 2020}
    \label{fig:space-gdp}
\end{figure}

Figure \ref{fig:space-gdp} shows the share of GDP spent on space activities in 2020. The United States, Russia, and France are the top three countries in terms of space spending as a share of GDP. The United States is the largest spender on space activities, accounting for 0.25\% of its GDP. Russia and France are the second and third largest spenders, accounting for 0.22\% and 0.12\% of their GDP, respectively.

\subsection{Global Space Interests - Trends and Forecasts}

\begin{figure}[h]
    \centering
    \includegraphics[width=0.8\textwidth]{Images/world-space-expenditure.png}
    \caption{Space Expenditures of countries in 2021}
    \label{fig:space-interests}
\end{figure}

Figure \ref{fig:space-interests} shows the space expenditures of countries in 2021 while Figure \ref{fig:korea-space-gdp} shows the gross domestic expenditure on space R\&D by sector in a developing country like Korea.

\begin{figure}[h]
    \centering
    \includegraphics[width=0.8\textwidth]{Images/korea-space-gdp.png}
    \caption{Gross domestic expenditure on space R\&D by sector in Korea}
    \label{fig:korea-space-gdp}
\end{figure}

Countries like India, China, and Korea are increasing their space spending, and the space economy is expected to grow significantly in the coming years. According to IN-SPace in this report \cite{news:thehindu}, the space economy of India is expected to grow to USD 44 billion by 2033, with about 8\% of global share.


\section{Observations}
These trends

\section{Conclusions}

\begin{thebibliography}{9}
    \bibitem{esa:space-economy1}
    \href{https://www.oecd.org/publications/oecd-handbook-on-measuring-the-space-economy-2nd-edition-8bfef437-en.htm}{OECD Handbook on Measuring the Space Economy}
    \bibitem{esa:space-economy2}
    What is the Space Economy? [Oct/2019] \href{https://space-economy.esa.int/article/33/what-is-the-space-economy#_ftn1}{European Space Agency}
    \bibitem{news:thehindu} IN-SPACe unveils decadal vision and strategy for Indian space economy [Oct 2023]
    \href{https://www.thehindu.com/sci-tech/science/indias-space-economy-has-potential-to-reach-35200-crore-44-billion-by-2033-with-about-8-of-global-share/article67403193.ece}{The Hindu}
\end{thebibliography}

\end{document}